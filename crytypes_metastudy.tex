\documentclass[11pt,twocolumn]{scrartcl}

\usepackage[margin=2cm]{geometry}
\setlength{\parskip}{1mm}
\setlength{\parindent}{10mm}

\title{\textbf{Metastudy on the state of the art of the classification of cry-types}}
\author{Franz Anders}
\date{}
\begin{document}

\maketitle
\tableofcontents

\section{Medical point of view}

\subsection{Crying as a Sign, a Sympton \& a Signal (2000)}

\subsubsection*{Summary:} 

It is \emph{not} possible to infer the cause by the cry characteristics. Crying is described as a "graded signal" wich expresesses the level of distress of the infant. In combination with contextual informations, the level of distress expressed can aid in identifying the cause.\cite{Gustafson2000}


\subsubsection*{Detailed:}

"We return now to the question asked at the outset: Are cry sounds of human infants unique to the eliciting condition - for example, hunger- pain- startle, or fatigue - and are they perceived uniformly and accuratly as such by their caregivers? Such notions are widely shared in the clinical, research and popular literatures on infant crying, and many parents and other caregivers believe the answer to be "Yes".  Nonethelesss, we have found little empirical support for such notions, and a great deal of evidence against them. The fundamental problem is that the sounds themselves appear not to be unitary and isomporphic with respect to discrete causes. [...]

Crying is different from other soundmaking and thus alerts the caregiver to the infant's distress. [...] Beyond alerting the caregiver, the sounds of crying convey level of distress or urgency of need. The probability of latency of a caregiver's response are thus affected. Level of distress per se appears to offer some clue as to the specific cause of crying. Coupled with contextual information - the infant's facial expression and bodily movements, surrounding events and caregiving schedule - the sound of crying may be highly informative with respect to a discrete cause. [...]

Nonetheless, it may be that not all crying has definable causes even with the help of contextual infromation. [...] Spock and Rothenberg(1985) note that almost all young infants get into "fretful periods" that cannot be precisely explained." \cite{Gustafson2000}

\subsection{Pain in Neonates and Infants (2007)}

\subsubsection*{Summary:} 
Cry offers information about the infants levels of distress.

\subsubsection*{Detailed:}

"Given there is no biologic gold standard for assessing pain in infants (Warnock and LAnder 2004), physiological, biobehavioural and behavioural indicators need to be considered as surrogate markers for self-report. [...]

Cry is a behavioural indicator of pain in infants that is frequently described in terms of presence or absence, temportal characteristics, amplitude and/or pitch. The temporal domain of cry includes latency to cry, duration of expiratory and inspirator5y cry, duratino of pause between cries and regulation or rythm of the cry bout. Amplitude is a measure of cry loudness, and pitch (measured as fundamental frequency) is the harmonic feature that includes variability, melody patterns, jitters, phonics and energy. Procedural-related cries are descrbied as intense and high-pitched, which signals the need for intervention. The cry occurs immediately or very shortly after the painful stimulus and can usually be distinguished from the cries due to hunger, anger or tiredness. Altough less understood, postoperative and chronic pain cries have been described as longer in duration and occuring with shorter latency than procedural-related cries.[...] 

Mature, healthy infants who are awake will elicit a robust cry in response to pain. Alternatively, preterm, acutely ill infants who are too immature or unwell may not cry and ventilated infants cannot cry in response to pain. Although cry can provide valuable information when it occurs, its absence cannot discount pain.[...]

Cues, such as an infant's cry, do not offer specific information about an infnt's pain per se but rather offer information about the intensity of the distress.
" 


\cite{Stevens2006}

\bibliography{crytypes_metastudy}
\bibliographystyle{alpha}

\end{document}
